\section{Introduction}

MARK II is simple SoC written in VHDL for synthetization into Altera Cyclone
FPGA. SoC consist of simple RISC like CPU and a few peripherals. There is also
small toolchain for writing programs that can run on MARK-II CPU.

CPU is 32 bits wide and following Von Neumann architecture, can address up to
$2^{24}$ words, has 32 interrupt sources, 16 registers including three special
ones.

To the CPU can be connected memory mapped peripherals, following peripherals
are available: UART, PS/2, VGA driver, general purpose timers, system timer,
GPIO, onchip RAM and ROM, interrupt controller.

Thanks to toolchain, you can write programms in human way. Toolchain consist of
Assembler, Linker, Emulator, Disassembler, Loader and ldm2mif converter.

This document is reference guide for MARK-II CPU. Please, note that this
document is not completed yet and some information can be outdated or even
missing! Every Time please read source codes or contact code authors.

MARK-II is home project, created just for fun. There isn't any main goals except
one, have fun from coding. As home fun project, is licensed under MIT license to
be free for everybody to try run it.

Anyway, some goals exist, but it's more like "I want to try ..." rather than
"I have to ...", these goals are changing in time and they are following topics
that I want to explore.
