\section{CPU core architecture}

\subsection{Registers}

\begin{table}[h]
    \centering
    \begin{tabular}{|l|l|l|l|}
        \hline
        \textbf{Register name} & \textbf{Purpose} & \textbf{Register name} & \textbf{Purpose} \\ \hline
        R0                     & zero register    & R8                     & 32b GPR          \\ \hline
        R1                     & 32b GPR          & R9                     & 32b GPR          \\ \hline
        R2                     & 32b GPR          & R10                    & 32b GPR          \\ \hline
        R3                     & 32b GPR          & R11                    & 32b GPR          \\ \hline
        R4                     & 32b GPR          & R12                    & 32b GPR          \\ \hline
        R5                     & 32b GPR          & R13                    & 32b GPR          \\ \hline
        R6                     & 32b GPR          & R14                    & Program counter  \\ \hline
        R7                     & 32b GPR          & R15                    & Stack pointer    \\ \hline
    \end{tabular}
    \caption{List of registers}
    \label{tab:registers_list}
\end{table}

In the table \ref{tab:registers_list} are listed all CPU registers, including
special ones.

GPR mean general purpose register, these registers are 32bit wide.
Zero register is one of special registers, it always contain zero.
You can write there whatever you want, but always read zero. Program
counter (PC) and Stack pointer (SP) are implemented like any others
registers but they are holding actual address in program and actual
address of top of stack.

There is no limitation in register usage in instructions. Every
instruction can work with every register include PC and SP. There is
no register windows, banks or something like that.

\subsection{Instruction set architecture}

There is only quick explanation of meaning individual instruction. If you want
to see binary format, see isa.ods file in doc folder.

\subsubsection{Program control instructions}

These instruction control program flow. In this group are instruction RET,
RETI, CALLI, BZI, BNZI, CALL, BZ and BNZ.

Instruction RETI is intended for returning from interrupt service routine. It
is necessary inform interrupt driver about completing the ISR. So, this is
reason why there is RETI.

Instruction RET, CALL and CALLI are for implementing subroutines. Instruction
CALL is similar to CALLI, both call subroutine and store return address into
stack. But instruction CALL excepting absolute address of subroutine.
Instruction CALLI, on the another hand, excepting register where the subroutine
address is stored. Instruction RET will take return address from stack and
continue normal program execution.

There are also instruction for conditional jumps. They are BZI, BNZI, BZ and
BNZ. BZ mean "branch zero" and BNZ mean "branch nonzero", suffix I mean where
address to jump is stored. If there is I in instruction, address is stored in
some register, instruction without I have address in their instruction word.

There isn't any flags for conditional jumps, for example, instruction BZ jump
when in specified register is zero, if there is another value, it will continue
without jumping.

\subsubsection{Memory instructions}

These instruction work with memory. They are LD, LDI, ST, STI, PUSH and POP.

PUSH and POP are instruction for stack operation. PUSH save data from specified
register into stack and decrement stack pointer. POP instruction increment
stack pointer and then store data from stack into specified register.

Instruction LD and LDI load data from memory into specified register. Memory is
organized as 32b words and these instruction work only with 32b words.
Instruction LD have address in it's instruction word and LDI use address from
specified register.

Instruction ST and STI are similar to LD and LDI, but they are load data into
memory.

\subsubsection{Data move instructions}

Instruction that are intended for moving data. There are only four instruction
of this type. First is MOV. MOV instruction move data from one register into
another. Actually, copy from one register into another.

Another two instruction from this group are MVIH and MVIL. Both are intended
for loading immediate values into register, but because there are 32bit
register, instruction are 32bit wide and there have to be space for opcode in
instruction word, MVI instruction is splitted into MVIH and MVIL. MVIL will
load lower 16b into specified register and MVIH upper 16bit.

Last instruction for moving is MVIA. MVIA is similar to MVIL and MVIH, but
constant for loading is 24bit wide. Because address bus of MARK II CPU is 24bit
wide, this instruction is perfect for loading address into registers.

\subsubsection{Computation instructions}

These instruction are intended for computation on the data. MARK II CPU use
three operand arithmetic, so you have to specify three register, two for
operand, and one for result. But, there are also few instruction where only one
operand is used, for example increment instruction.

CPU is able to do basic bitwise operations, concretely AND, OR, XOR. There are
also ADD and SUB instruction for adding and subtracting, be ware there is no
overflow control.

Instructions IND and DEC instructions for incrementing and decrementing
register, these instruction use only two operand, one for input operand and
second for result.

There are also shifting instructions LSL, LSR, ROL and ROR. LSL for logical
shift to left, LSR for logical shift to right, ROL for rotate to left and ROR
for rotate to right. You can also specify distance from 0 to 15. Shift and
rotations are done in barrel shifter.

\subsubsection{Special instructions}

Currently there is only one special instruction and that is CMP. This
instruction is mandatory for branching. With this instruction you can compare
two register with specified comparison type and store 1 if "expression" is true
or 0 if false, into specified register.

\subsubsection{List of instructions}

\begin{itemize}

    \item RET
    \begin{itemize}
        \item \textbf{Syntax}
        \begin{lstlisting}[language={[x86masm]Assembler}, frame=single]
    RET
        \end{lstlisting}
        \item \textbf{Explanation}
        \begin{lstlisting}[language=bash, frame=single]
    SP++; PC <= mem(SP)
        \end{lstlisting}
        \item \textbf{Comment} \\
    Return from subroutine.
    \end{itemize}

    \item RETI
    \begin{itemize}
        \item \textbf{Syntax}
        \begin{lstlisting}[language={[x86masm]Assembler}, frame=single]
    RETI
        \end{lstlisting}
        \item \textbf{Explanation}
        \begin{lstlisting}[language=bash, frame=single]
    SP++; PC <= mem(SP)
        \end{lstlisting}
        \item \textbf{Comment} \\
    Return from interrupt service routine.
    \end{itemize}

    \item CALLI
    \begin{itemize}
        \item \textbf{Syntax}
        \begin{lstlisting}[language={[x86masm]Assembler}, frame=single]
    CALLI rA
        \end{lstlisting}
        \item \textbf{Explanation}
        \begin{lstlisting}[language=bash, frame=single]
    mem(SP) <= PC; SP--; PC <= reg(rA)
        \end{lstlisting}
        \item \textbf{Comment} \\
    Call subroutine with address in some of register.
    \end{itemize}

    \item PUSH
    \begin{itemize}
        \item \textbf{Syntax}
        \begin{lstlisting}[language={[x86masm]Assembler}, frame=single]
    PUSH rA
        \end{lstlisting}
        \item \textbf{Explanation}
        \begin{lstlisting}[language=bash, frame=single]
    mem(SP) <= reg(rA); SP--
        \end{lstlisting}
        \item \textbf{Comment} \\
    Store register into stack, then decrement stack pointer.
    \end{itemize}

    \item POP
    \begin{itemize}
        \item \textbf{Syntax}
        \begin{lstlisting}[language={[x86masm]Assembler}, frame=single]
    POP rA
        \end{lstlisting}
        \item \textbf{Explanation}
        \begin{lstlisting}[language=bash, frame=single]
    SP++ reg(rA) <= mem(SP)
        \end{lstlisting}
        \item \textbf{Comment} \\
    Get data from stack and store them in register, then increment stack pointer.
    \end{itemize}

    \item LDI
    \begin{itemize}
        \item \textbf{Syntax}
        \begin{lstlisting}[language={[x86masm]Assembler}, frame=single]
    LDI rA rB
        \end{lstlisting}
        \item \textbf{Explanation}
        \begin{lstlisting}[language=bash, frame=single]
    reg(rB) <= mem(reg(rA))
        \end{lstlisting}
        \item \textbf{Comment} \\
    Load from memory. Address is stored in some of register.
    \end{itemize}

    \item STI
    \begin{itemize}
        \item \textbf{Syntax}
        \begin{lstlisting}[language={[x86masm]Assembler}, frame=single]
    STI rA rB
        \end{lstlisting}
        \item \textbf{Explanation}
        \begin{lstlisting}[language=bash, frame=single]
    mem(reg(rB)) <= reg(rA)
        \end{lstlisting}
        \item \textbf{Comment} \\
    Store data from register into memory. Address is stored in some of register.
    \end{itemize}

    \item BZI
    \begin{itemize}
        \item \textbf{Syntax}
        \begin{lstlisting}[language={[x86masm]Assembler}, frame=single]
    BZI rA rB
        \end{lstlisting}
        \item \textbf{Explanation}
        \begin{lstlisting}[language=bash, frame=single]
    if reg(rA) = 0 then PC <= reg(rB)
        \end{lstlisting}
        \item \textbf{Comment} \\
    Branch if register is equal to zero. Address for jump is stored in register.
    \end{itemize}

    \item BNZI
    \begin{itemize}
        \item \textbf{Syntax}
        \begin{lstlisting}[language={[x86masm]Assembler}, frame=single]
    BNZI rA rB
        \end{lstlisting}
        \item \textbf{Explanation}
        \begin{lstlisting}[language=bash, frame=single]
    if reg(rA) != 0 then PC <= reg(rB)
        \end{lstlisting}
        \item \textbf{Comment} \\
    Branch if register isn't equal to zero. Address for jump is stored in register.
    \end{itemize}

    \item MOV
    \begin{itemize}
        \item \textbf{Syntax}
        \begin{lstlisting}[language={[x86masm]Assembler}, frame=single]
    MOV rA rB
        \end{lstlisting}
        \item \textbf{Explanation}
        \begin{lstlisting}[language=bash, frame=single]
    reg(rB) <= reg(rA)
        \end{lstlisting}
        \item \textbf{Comment} \\
    Move data from one register to the another.
    \end{itemize}

    \item CMP
    \begin{itemize}
        \item \textbf{Syntax}
        \begin{lstlisting}[language={[x86masm]Assembler}, frame=single]
    CMP Comp rA rB rC
        \end{lstlisting}
        \item \textbf{Explanation}
        \begin{lstlisting}[language=bash, frame=single]
    if reg(rA) "Comp" reg(rB) then reg(rC) <= 1
        \end{lstlisting}
        \item \textbf{Comment} \\
    Compare two registers. Condition is "Comp" and can be found in table \ref{tab:cmp_conds}.

    \begin{table}[h]
        \centering
        \begin{tabular}{|l|l|l|}
            \hline
            \textbf{Name}             & \textbf{Symbol} & \textbf{Meaning}      \\ \hline
            Equal                     & EQ              & $A==B$                \\ \hline
            Not Equal                 & NQE             & $A!=B$                \\ \hline
            Lower than                & L               & $A<B$                 \\ \hline
            Lower than (U)            & LU              & $A<B$ Unsigned        \\ \hline
            Greater or equal than     & GE              & $A>B | A==B$          \\ \hline
            Greater or equal than (U) & GEU             & $A>B | A==B$ Unsigned \\ \hline
        \end{tabular}
        \caption{Conditions for CMP instruction.}
        \label{tab:cmp_conds}
    \end{table}

    \end{itemize}

    \item AND
    \begin{itemize}
        \item \textbf{Syntax}
        \begin{lstlisting}[language={[x86masm]Assembler}, frame=single]
    AND rA rB rC
        \end{lstlisting}
        \item \textbf{Explanation}
        \begin{lstlisting}[language=bash, frame=single]
    reg(rC) <= reg(rA) AND reg(rB)
        \end{lstlisting}
        \item \textbf{Comment} \\
    Bitwise AND between two registers.
    \end{itemize}

    \item OR
    \begin{itemize}
        \item \textbf{Syntax}
        \begin{lstlisting}[language={[x86masm]Assembler}, frame=single]
    OR rA rB rC
        \end{lstlisting}
        \item \textbf{Explanation}
        \begin{lstlisting}[language=bash, frame=single]
    reg(rC) <= reg(rA) OR reg(rB)
        \end{lstlisting}
        \item \textbf{Comment} \\
    Bitwise OR between two registers.
    \end{itemize}

    \item XOR
    \begin{itemize}
        \item \textbf{Syntax}
        \begin{lstlisting}[language={[x86masm]Assembler}, frame=single]
    XOR rA rB rC
        \end{lstlisting}
        \item \textbf{Explanation}
        \begin{lstlisting}[language=bash, frame=single]
    reg(rC) <= reg(rA) XOR reg(rB)
        \end{lstlisting}
        \item \textbf{Comment} \\
    Bitwise XOR between two registers.
    \end{itemize}

    \item ADD
    \begin{itemize}
        \item \textbf{Syntax}
        \begin{lstlisting}[language={[x86masm]Assembler}, frame=single]
    ADD rA rB rC
        \end{lstlisting}
        \item \textbf{Explanation}
        \begin{lstlisting}[language=bash, frame=single]
    reg(rC) <= reg(rA) + reg(rB)
        \end{lstlisting}
        \item \textbf{Comment} \\
    32 bits wide addition.
    \end{itemize}

    \item SUB
    \begin{itemize}
        \item \textbf{Syntax}
        \begin{lstlisting}[language={[x86masm]Assembler}, frame=single]
    SUB rA rB rC
        \end{lstlisting}
        \item \textbf{Explanation}
        \begin{lstlisting}[language=bash, frame=single]
    reg(rC) <= reg(rA) - reg(rB)
        \end{lstlisting}
        \item \textbf{Comment} \\
    32 bits wide subtraction.
    \end{itemize}

    \item INC
    \begin{itemize}
        \item \textbf{Syntax}
        \begin{lstlisting}[language={[x86masm]Assembler}, frame=single]
    INC rA rB
        \end{lstlisting}
        \item \textbf{Explanation}
        \begin{lstlisting}[language=bash, frame=single]
    reg(rB) <= reg(rA) + 1
        \end{lstlisting}
        \item \textbf{Comment} \\
    Increment value from register and store it into another.
    \end{itemize}

    \item DEC
    \begin{itemize}
        \item \textbf{Syntax}
        \begin{lstlisting}[language={[x86masm]Assembler}, frame=single]
    DEC rA rB
        \end{lstlisting}
        \item \textbf{Explanation}
        \begin{lstlisting}[language=bash, frame=single]
    reg(rB) <= reg(rA) - 1
        \end{lstlisting}
        \item \textbf{Comment} \\
    Decrement value from register and store it into another.
    \end{itemize}

    \item LSL
    \begin{itemize}
        \item \textbf{Syntax}
        \begin{lstlisting}[language={[x86masm]Assembler}, frame=single]
    LSL Dist rA rB
        \end{lstlisting}
        \item \textbf{Explanation}
        \begin{lstlisting}[language=bash, frame=single]
    reg(rB) <= reg(rA) left shift by "Dist"
        \end{lstlisting}
        \item \textbf{Comment} \\
    Logical shift to left using barell shifter. "Dist" should be number from $0$ to $15$.
    \end{itemize}

    \item LSR
    \begin{itemize}
        \item \textbf{Syntax}
        \begin{lstlisting}[language={[x86masm]Assembler}, frame=single]
    LSR Dist rA rB
        \end{lstlisting}
        \item \textbf{Explanation}
        \begin{lstlisting}[language=bash, frame=single]
    reg(rB) <= reg(rA) righ shift by "Dist"
        \end{lstlisting}
        \item \textbf{Comment} \\
    Logical shift to right using barell shifter. "Dist" should be number from $0$ to $15$.
    \end{itemize}

    \item ROL
    \begin{itemize}
        \item \textbf{Syntax}
        \begin{lstlisting}[language={[x86masm]Assembler}, frame=single]
    ROL Dist rA rB
        \end{lstlisting}
        \item \textbf{Explanation}
        \begin{lstlisting}[language=bash, frame=single]
    reg(rB) <= reg(rA) left rotate by "Dist"
        \end{lstlisting}
        \item \textbf{Comment} \\
    Rotate to left using barell shifter. "Dist" should be number from $0$ to $15$.
    \end{itemize}

    \item ROR
    \begin{itemize}
        \item \textbf{Syntax}
        \begin{lstlisting}[language={[x86masm]Assembler}, frame=single]
    ROR Dist rA rB
        \end{lstlisting}
        \item \textbf{Explanation}
        \begin{lstlisting}[language=bash, frame=single]
    reg(rB) <= reg(rA) righ rotate by "Dist"
        \end{lstlisting}
        \item \textbf{Comment} \\
    Rotate to right using barell shifter. "Dist" should be number from $0$ to $15$.
    \end{itemize}

    \item MVIL
    \begin{itemize}
        \item \textbf{Syntax}
        \begin{lstlisting}[language={[x86masm]Assembler}, frame=single]
    MVIL rA Cons16
        \end{lstlisting}
        \item \textbf{Explanation}
        \begin{lstlisting}[language=bash, frame=single]
    reg(rA) <= (reg(rA) AND 0xFFFF0000) OR "Cons16"
        \end{lstlisting}
        \item \textbf{Comment} \\
    Move immediate data (stored in instruction word) into lower half of register.
    \end{itemize}

    \item MVIH
    \begin{itemize}
        \item \textbf{Syntax}
        \begin{lstlisting}[language={[x86masm]Assembler}, frame=single]
    MVIH rA cons16
        \end{lstlisting}
        \item \textbf{Explanation}
        \begin{lstlisting}[language=bash, frame=single]
    reg(rA) <= (reg(rA) AND 0x0000FFFF) OR ("Cons16" << 16)
        \end{lstlisting}
        \item \textbf{Comment} \\
    Move immediate data (stored in instruction word) into higher half of register.
    \end{itemize}

    \item CALL
    \begin{itemize}
        \item \textbf{Syntax}
        \begin{lstlisting}[language={[x86masm]Assembler}, frame=single]
    CALL Addr
        \end{lstlisting}
        \item \textbf{Explanation}
        \begin{lstlisting}[language=bash, frame=single]
    mem(SP) <= PC; SP--; PC <= Addr
        \end{lstlisting}
        \item \textbf{Comment} \\
    Call subroutine and store PC into SP. Address of subroutine is in instruction.
    \end{itemize}

    \item LD
    \begin{itemize}
        \item \textbf{Syntax}
        \begin{lstlisting}[language={[x86masm]Assembler}, frame=single]
    LD Addr rA
        \end{lstlisting}
        \item \textbf{Explanation}
        \begin{lstlisting}[language=bash, frame=single]
    reg(rA) <= mem(Addr)
        \end{lstlisting}
        \item \textbf{Comment} \\
    Load data from memory into register. Address is in instruction.
    \end{itemize}

    \item ST
    \begin{itemize}
        \item \textbf{Syntax}
        \begin{lstlisting}[language={[x86masm]Assembler}, frame=single]
    ST rA Addr
        \end{lstlisting}
        \item \textbf{Explanation}
        \begin{lstlisting}[language=bash, frame=single]
    mem(Addr) <= reg(rA)
        \end{lstlisting}
        \item \textbf{Comment} \\
    Store data from register into memory. Address is in instruction.
    \end{itemize}

    \item BZ
    \begin{itemize}
        \item \textbf{Syntax}
        \begin{lstlisting}[language={[x86masm]Assembler}, frame=single]
    BZ rA Addr
        \end{lstlisting}
        \item \textbf{Explanation}
        \begin{lstlisting}[language=bash, frame=single]
    if reg(rA) = 0 then PC <= Addr
        \end{lstlisting}
        \item \textbf{Comment} \\
    Branch if register is equal to zero. Address for jump is stored in instruction.
    \end{itemize}

    \item BNZ
    \begin{itemize}
        \item \textbf{Syntax}
        \begin{lstlisting}[language={[x86masm]Assembler}, frame=single]
    BNZ rA Addr
        \end{lstlisting}
        \item \textbf{Explanation}
        \begin{lstlisting}[language=bash, frame=single]
    if reg(rA) != 0 then PC <= Addr
        \end{lstlisting}
        \item \textbf{Comment} \\
    Branch if register isn't equal to zero. Address for jump is stored in register.
    \end{itemize}

    \item MVIA
    \begin{itemize}
        \item \textbf{Syntax}
        \begin{lstlisting}[language={[x86masm]Assembler}, frame=single]
    MVIA rA Addr
        \end{lstlisting}
        \item \textbf{Explanation}
        \begin{lstlisting}[language=bash, frame=single]
    reg(rA) <= Addr
        \end{lstlisting}
        \item \textbf{Comment} \\
    Load 24b constant into specified register. This is usefull for loading label addresses.
    \end{itemize}

\end{itemize}

\subsection{Memory organization}

CPU can address up to $2^{24}$ words, whole memory space is linear. Everything
is in one memory space, there is nothing like IO space, program space, data
space. MARK II CPU following Von Neumann scheme, so program and also data is in
one space. Peripherals and IO devices should be mapped there too.

There are two instruction for working with memory. One is LD for reading from
memory, and second is ST for writing into memory. Both instruction can use
absolute addressing mode (address is part of instruction) or indirect
addressing with register (address is stored in register).

\subsubsection{Stack}

Stack is used for storing returning address when calling subroutines or
interrupts. It grow from up to down. So, you should set SP at end of ram.

You can also use stack for storing your variables/registers etc. There is two
special functions for that.

\begin{itemize}
    \item \textbf{PUSH} - store data into stack and decrement SP
    \item \textbf{POP} - take data from stack and increment SP
\end{itemize}

\subsection{Interrupts}

Interrupt vectors are address in memory where the CPU will jump when interrupt
occur. All addresses can be found in the table \ref{tab:intsources}. When
interrupt occur CPU will do following:

1. complete actual instruction
2. increment PC (PC always have to point to next instruction)
3. store PC into stack and decrement SP
4. jump into interrupt vector

There is only two words for every vector. This is enough only for code like this:

\begin{lstlisting}[language={[x86masm]Assembler}, frame=single]
    .ORG UART1_RX_VECTOR
        CALL UART1_RX_ISR
        RETI
\end{lstlisting}

Instruction RETI is mandatory. It will inform interrupt driver that ISR is
completed and CPU is ready to take another interrupt. There is no support for
nested interrupts. When first interrupt come, another have to wait.

