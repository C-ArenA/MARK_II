\subsection{Registers}

In the table \ref{tab:registers_list} are listed all CPU registers, including
special ones.

GPR mean general purpose register, these registers are 32bit wide.
Zero register is one of special registers, it always contain zero.
You can write there whatever you want, but always read zero. Program
counter (PC) and Stack pointer (SP) are implemented like any others
registers but they are holding actual address in program and actual
address of top of stack.

There is no limitation in register usage in instructions. Every
instruction can work with every register include PC and SP. There is
no register windows, banks or something like that.

\begin{table}[h]
    \centering
    \begin{tabular}{|l|l|l|l|}
        \hline
        \textbf{Register name} & \textbf{Purpose} & \textbf{Register name} & \textbf{Purpose} \\ \hline
        R0                     & zero register    & R8                     & 32b GPR          \\ \hline
        R1                     & 32b GPR          & R9                     & 32b GPR          \\ \hline
        R2                     & 32b GPR          & R10                    & 32b GPR          \\ \hline
        R3                     & 32b GPR          & R11                    & 32b GPR          \\ \hline
        R4                     & 32b GPR          & R12                    & 32b GPR          \\ \hline
        R5                     & 32b GPR          & R13                    & 32b GPR          \\ \hline
        R6                     & 32b GPR          & R14                    & Program counter  \\ \hline
        R7                     & 32b GPR          & R15                    & Stack pointer    \\ \hline
    \end{tabular}
    \caption{List of registers}
    \label{tab:registers_list}
\end{table}


