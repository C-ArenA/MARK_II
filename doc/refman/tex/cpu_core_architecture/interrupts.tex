\subsection{Interrupts}

Interrupt vectors are address in memory where the CPU will jump when interrupt
occur. All addresses can be found in the table \ref{tab:intsources}. When
interrupt occur CPU will do following:

1. complete actual instruction
2. increment PC (PC always have to point to next instruction)
3. store PC into stack and decrement SP
4. jump into interrupt vector

There is only two words for every vector. This is enough only for code like this:

\begin{lstlisting}[language={[x86masm]Assembler}, frame=single]
    .ORG UART1_RX_VECTOR
        CALL UART1_RX_ISR
        RETI
\end{lstlisting}

Instruction RETI is mandatory. It will inform interrupt driver that ISR is
completed and CPU is ready to take another interrupt. There is no support for
nested interrupts. When first interrupt come, another have to wait.
