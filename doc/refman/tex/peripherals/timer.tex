\subsection{16bit Timer}

This timer is more complex than SysTimer, but is only 16bit wide. It 
can also generate interrupt on compare match and have pre-scaler. Clock 
is hard-wired to CPU clock.

\subsubsection{Function}

Timer are clocked at same frequency as CPU and this cannot be changed. It
is counting from zero to up, until compare match with specified top occur. Then
interrupt request is generated, counter is set to zero and counting are restarted.

Timer have two registers, one for controlling and one with actual counter value. Reading
from registers are possible, writing too but when write is performed at counter
value register, counting is restarted from zero.

With control register you can set top value, enable and disable timer or interrupt.

\subsubsection{Registers}

All registers are listed in table \ref{tab:timer_reg_map}. Register TVRx is 16
bits wide and  contains  actual value of counter. 

\begin{table}[h]
    \centering
    \begin{tabular}{|l|l|l|}
        \hline
        \textbf{Offset} & \textbf{Name} & \textbf{Purpose}            \\ \hline
        $+0$            & TCRx          & Timer X control register.    \\ \hline
        $+1$            & TVRx          & Timer X value register.      \\ \hline
    \end{tabular}
    \caption{16bit timer register map}
    \label{tab:timer_reg_map}
\end{table}

\begin{figure}[H]
    \centering
    \begin{tikzpicture}
        \bitrect{16}{32-\bit}        
        \robits{0}{11}{}
        \rwbits{11}{3}{DIV}
        \rwbits{14}{1}{ien}
        \rwbits{15}{1}{ten}
    \end{tikzpicture}
    \begin{tikzpicture}
        \bitrect{16}{16-\bit}
        \rwbits{0}{16}{CMPM}
    \end{tikzpicture}
    \caption{TCRx register}
    \label{fig:tcr_reg}
\end{figure}

Register TCRx is control register of Timer X. You can set CMPM value, 
to set top value of counting, you can enable whole timer here, enable 
interrupts and set pre-scaler. In TCRx bits have following meaning:

\begin{itemize}
	\item \textbf{CMPM} 16bit TOP value
	\item \textbf{ten} enable timer
	\item \textbf{ien} enable interrupt
	\item \textbf{DIV} 3bit wide prescaler select
\end{itemize}

All possible configurations for pre-scaler are listed in table \ref{tab:tim_pres}.

\begin{table}[H]
    \centering
    \begin{tabular}{|l|l|l|l|}
        \hline
        \textbf{$DIV_2$} & \textbf{$DIV_1$} & \textbf{$DIV_0$} & \textbf{pre-scaler} \\ \hline
        0 & 0 & 0 & $F_{clk}$ \\ \hline
		0 & 0 & 1 & $\frac{1}{2} * F_{CLK}$  \\ \hline        
		0 & 1 & 0 & $\frac{1}{4} * F_{CLK}$  \\ \hline        
		0 & 1 & 1 & $\frac{1}{8} * F_{CLK}$  \\ \hline
		1 & 0 & 0 & $\frac{1}{16} * F_{CLK}$ \\ \hline
		1 & 0 & 1 & \textit{reserved}    \\ \hline        
		1 & 1 & 0 & \textit{reserved}    \\ \hline        
		1 & 1 & 1 & \textit{reserved}    \\ \hline        
    \end{tabular}
    \caption{16bit timer pre-scaler}
    \label{tab:tim_pres}
\end{table}

\subsubsection{Hacking}

One can simply add another instances of 16bit timer if needed. 
Interface consist from bus interface and one interrupt request signal. 
When you wand add another timer, just remember to connect intrq to the 
interrupt controller.

\begin{lstlisting}[language=VHDL, frame=single]
entity timer is
    generic(
        BASE_ADDRESS: unsigned(23 downto 0) := x"000000"
    );
    port(
        clk: in std_logic;
        res: in std_logic;
        address: in std_logic_vector(23 downto 0);
        data_mosi: in std_logic_vector(31 downto 0);
        data_miso: out std_logic_vector(31 downto 0);
        WR: in std_logic;
        RD: in std_logic;
        ack: out std_logic;
        intrq: out std_logic
    );
end entity timer;
\end{lstlisting}

