\subsection{Interrupt driver}

Interrupt driver is one of peripherals tightly connected to the CPU. It is
responsible for taking care about interrupts. It can disable some interrupt
vectors, it also have to decide what interrupt should be evaluated when more
request come at once. Also, interrupt driver have to take care about interrupt
queue where interrupt requests are waiting for CPU time. And finally, interrupt
driver maintain addresses of ISR.

\subsubsection{Function}

Interrupt driver is able to handle up to 16 interrupt sources. Each source have
to be enabled in interrupt mask register (INTMR) and its ISR address have to be
set.

Please note that some peripherals have to be configured to enable own
interrupts too.

There is totaly 17 registers in interrupt driver. One is INTMR. This register
is 16 bit wide and is used to enabling/disabling interrupt sources. In this
register bit no. 0 is controlling interrupt source 0, bit 1 interrupt source 1
and so on.

Priority of individuals interrupts are decreasing with interrupt number, for
example interrupt 0 is smallest number, so it have highest priority, interrupt
15 is largest interrupt number, so have lowest priority.

Additional 16 registers are used to specify address of individuals interrupt
service routines. One register for one interrupt source. There registers are
24 bit wides due to wide of address bus.

If one want enable interrupt from specified source, following have to be done:

\begin{itemize}
    \item Configure peripheral for generating interrupt (see peripheral chapter)
    \item Set address of interrupt service routine to appropriate register
    \item Write log. 1 to appropriate bit of INTMR
\end{itemize}


\subsubsection{Registers}

All registers are listed in table \ref{tab:int_driver_reg_map}.

\begin{table}[H]
    \centering
    \begin{tabular}{|l|l|l|}
        \hline
        \textbf{Offset} & \textbf{Name} & \textbf{Purpose}             \\ \hline
        $+0$            & INTMR         & Mask for enabling interrupts \\ \hline
        $+1$            & INTVEC0       & Address of ISR0              \\ \hline
        $+2$            & INTVEC1       & Address of ISR1              \\ \hline
        $+3$            & INTVEC2       & Address of ISR2              \\ \hline
        $+4$            & INTVEC3       & Address of ISR3              \\ \hline
        $+5$            & INTVEC4       & Address of ISR4              \\ \hline
        $+6$            & INTVEC5       & Address of ISR5              \\ \hline
        $+7$            & INTVEC6       & Address of ISR6              \\ \hline
        $+8$            & INTVEC7       & Address of ISR7              \\ \hline
        $+9$            & INTVEC8       & Address of ISR8              \\ \hline
        $+10$           & INTVEC9       & Address of ISR9              \\ \hline
        $+11$           & INTVEC10      & Address of ISR10             \\ \hline
        $+12$           & INTVEC11      & Address of ISR11             \\ \hline
        $+13$           & INTVEC12      & Address of ISR12             \\ \hline
        $+14$           & INTVEC13      & Address of ISR13             \\ \hline
        $+15$           & INTVEC14      & Address of ISR14             \\ \hline
        $+16$           & INTVEC15      & Address of ISR15             \\ \hline
    \end{tabular}
    \caption{Interrupt driver registers map}
    \label{tab:int_driver_reg_map}
\end{table}

\subsubsection{Hacking}

Interrupt driver can't be modified because it is highly coupled with CPU. When
you want to modify SoC, you have every time had interrupt controlled in your design.

There is of course standard bus interface for this peripheral, and some device
specific signals. These are int\_req, int\_accept, int\_completed,
int\_cpu\_address and int\_cpu\_rq.

Signals int\_accept, int\_completed, int\_cpu\_address and int\_cpu\_rq should
be connected directly to the CPU and are used for communication between CPU and
interrupt driver.

Signal int\_req is 16 bits wide vector of interrupt request signals. These should be
connected to the peripherals interrupt request outputs.

\begin{lstlisting}[language=VHDL, frame=single]
entity intController is
    generic(
        BASE_ADDRESS: unsigned(23 downto 0) := x"00000"
    );
    port(
        clk: in std_logic;
        res: in std_logic;
        address: in std_logic_vector(23 downto 0);
        data_mosi: in std_logic_vector(31 downto 0);
        data_miso: out std_logic_vector(31 downto 0);
        WR: in std_logic;
        RD: in std_logic;
        ack: out std_logic;
        int_req: in std_logic_vector(15 downto 0);
        int_accept: in std_logic;
        int_completed: in std_logic;
        int_cpu_address: out std_logic_vector(23 downto 0);
        int_cpu_rq: out std_logic
    );
end entity intController;
\end{lstlisting}

