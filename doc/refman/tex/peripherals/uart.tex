\subsection{UART}

UART is simple way to communicate with another microprocessor system or
computer. UART core in MARK II SoC have build-in baudrate generator, FIFO
buffers for transmitting and receiving, and six configurable interrupt
conditions. However, communication format is hardwired to 8N1.

\subsubsection{Function}

From the CPU side the UART is seen as four registers. There registers are:

\begin{itemize}
    \item Control register
    \item Status register
    \item TX data register
    \item RX data register
\end{itemize}

All registers are described detailed in section \ref{sec:peripherals_uart_reg}.
TX register and RX register is not actually register. Actually these locations
are connected to RX/TX FIFO. But you can use them like register. Both FIFO buffers
are able to store up to 32 bytes.

When you want to use UART you have to configure it first. This can be done in
control register. You have to enable transmitter and/or receiver. Then you have
to specify $n$ constant for baudrate generator.

Constant $n$ is integer constant that control baudrate generator. It can be
calculated using following formula:

$$
    n = \frac{F_{clk_uart}}{baudrate*16} - 1
$$

Optionally, you may want to enable interrupts.

When configuration is done, then you can start transmitting. For doing so,
simply write byte to send into TX data register, transmit will be triggered
automatically, and will continue until there are something left in FIFO buffer.

Interrupt may be generated when TX FIFO buffer is empty, half empty, or even when
one byte is sent.

Receiving is fully automatic. When start condition occur on RX line, receiving is
triggered. After whole byte is received is automatically stored into RX FIFO where
data waiting for CPU to be read.

Interrupt may be generated when new byte is received, or RX FIFO buffer is full, or even
half full.

When some interrupt is raised, flag in status register is also set. Flags in this
register can be used to determine, what kind of interrupt request was raised. Flags
can be read only once and cannot be written. Also, when flag is active, another interrupt
request cannot be raised. So it is important to read status register in ISR.

In status register are also current counts of bytes stored in buffers.

\subsubsection{Registers}
\label{sec:peripherals_uart_reg}

Table \ref{tab:uart_reg_map} refer all registers with their offsets. Bit meaning
of individual registers are on figures \ref{fig:UTDR_reg} to \ref{fig:USR_reg}.

\begin{table}[H]
    \centering
    \begin{tabular}{|l|l|l|}
        \hline
        \textbf{Offset} & \textbf{Name} & \textbf{Purpose}           \\ \hline
        $+0$            & UTDR          & Transmitter data register  \\ \hline
        $+1$            & URDR          & Receiver data register     \\ \hline
        $+2$            & USR           & Status register            \\ \hline
        $+3$            & UCR           & Control register           \\ \hline
    \end{tabular}
    \caption{UART register map}
    \label{tab:uart_reg_map}
\end{table}

\begin{figure}[H]
    \centering
    \begin{tikzpicture}
        \bitrect{16}{32-\bit}
        \robits{0}{16}{}
    \end{tikzpicture}
    \begin{tikzpicture}
        \bitrect{16}{16-\bit}
        \robits{0}{8}{}
        \rwbits{8}{8}{Tx data}
    \end{tikzpicture}
    \caption{UTDR register}
    \label{fig:UTDR_reg}
\end{figure}

\begin{figure}[H]
    \centering
    \begin{tikzpicture}
        \bitrect{16}{32-\bit}
        \robits{0}{16}{}
    \end{tikzpicture}
    \begin{tikzpicture}
        \bitrect{16}{16-\bit}
        \robits{0}{8}{}
        \rwbits{8}{8}{Rx data}
    \end{tikzpicture}
    \caption{URDR register}
    \label{fig:URDR_reg}
\end{figure}

\begin{figure}[H]
    \centering
    \begin{tikzpicture}
        \bitrect{16}{32-\bit}
        \rwbits{15}{1}{rxen}
        \rwbits{14}{1}{txen}
        \rwbits{13}{1}{inten}
        \rwbits{12}{1}{rfint}
        \rwbits{11}{1}{rhint}
        \rwbits{10}{1}{rxint}
        \rwbits{9}{1}{teint}
        \rwbits{8}{1}{thint}
        \rwbits{7}{1}{txint}
        \robits{0}{7}{}
    \end{tikzpicture}
    \begin{tikzpicture}
        \bitrect{16}{16-\bit}
        \rwbits{0}{16}{N}
    \end{tikzpicture}
    \caption{UCR register}
    \label{fig:UCR_reg}
\end{figure}

UCR register is control register. There is all control bits to configure UART interface.
Meaning of individual bits is in following list:

\begin{itemize}
    \item \textbf{rxen} - Enable receiver
    \item \textbf{txen} - Enable transmitting
    \item \textbf{inten} - Enable interrupt
    \item \textbf{rfint} - Enable interrupt full RX FIFO
    \item \textbf{rhint} - Enable interrupt half full RX FIFO
    \item \textbf{rxint} - Enable interrupt RX byte come
    \item \textbf{teint} - Enable interrupt empty TX FIFO
    \item \textbf{thint} - Enable interrupt half empty TX FIFO
    \item \textbf{txint} - Enable interrupt TX byte sent
\end{itemize}

Bit field called $N$ is constant for baudrate generator.

\begin{figure}[H]
    \centering
    \begin{tikzpicture}
        \bitrect{16}{32-\bit}
        \robits{0}{14}{}
        \rwbits{15}{1}{thif}
        \rwbits{14}{1}{txif}
    \end{tikzpicture}
    \begin{tikzpicture}
        \bitrect{16}{16-\bit}
        \rwbits{10}{6}{rxcount}
        \rwbits{4}{6}{txcount}
        \rwbits{3}{1}{rfif}
        \rwbits{2}{1}{rhif}
        \rwbits{1}{1}{rxif}
        \rwbits{0}{1}{teif}
    \end{tikzpicture}
    \caption{USR register}
    \label{fig:USR_reg}
\end{figure}

Register USR can be read but cannot be set. It contain status of UART interface.
There are two bit field, rxcount and txcount, that inform about state of FIFO buffers.
There is also six flags that can be used to determinate what kind of interrupt
was raised. By reading these flags is cleared. Individual flags are described in
following list:

\begin{itemize}
    \item \textbf{rfif} - full RX FIFO interrupt flag
    \item \textbf{rhif} - half full RX FIFO interrupt flag
    \item \textbf{rxif} - RX byte come interrupt flag
    \item \textbf{teif} - empty TX FIFO interrupt flag
    \item \textbf{thif} - half empty TX FIFO interrupt flag
    \item \textbf{txif} - TX byte sent interrupt flag
\end{itemize}

\subsubsection{Hacking}

UART peripherals does not have ability to change it parameters simply. They are
all hardwired. Anyway, you may want add more UART units, that is of course
possible.

\begin{lstlisting}[language=VHDL, frame=single]
entity uart is
    generic(
        BASE_ADDRESS: unsigned(23 downto 0) := x"000000"
    );
    port(
        clk: in std_logic;
        res: in std_logic;
        address: in unsigned(23 downto 0);
        data_mosi: in unsigned(31 downto 0);
        data_miso: out unsigned(31 downto 0);
        WR: in std_logic;
        RD: in std_logic;
        ack: out std_logic;
        --device
        clk_uart: in std_logic;
        rx: in std_logic;
        tx: out std_logic;
        intrq: out std_logic
    );
end entity uart;
\end{lstlisting}

Entity UART have same bus interface as all others modules. That mean clk, res,
address, data\_mosi, data\_miso, WR, RD, ack and BASE\_ADDRESS argument. But there
are also some device specific signals. These are rx, tx, intrq and clk\_uart.

Signals intrq is interrupt requests. You should connect it to
the input of the interrupt controller. Signals rx and tx are transmitter and
receiver signals. You probably want connect them to the top levels pins. If rx
pin is not connected, there should be logical one on it.

Signal clk\_uart is clock for baudrate generator. It should be connected to PLL
component, that generate desired clock for uarts.
