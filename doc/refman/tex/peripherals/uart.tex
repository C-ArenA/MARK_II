\subsection{UART}

This is an simple full duplex UART driver, hardwired to 8N1 format, with
ability to change baud-rate and with interrupt on RX and TX events.

\subsubsection{Function}

UART have three registers. First is control register, where constant called n
is stored, this constant is 16bit wide and it is controlling the baud-rate
generator. Value of constant n can be calculated like this:

$$
    n = \frac{F_{clk}}{baudrate*16} - 1
$$

But remember, only integer constant can be stored in this register. Some
prepared constant value can be found in /doc/baudrate.ods file.

Next two registers are RX register and TX register. They both have same
address, but when you read from this address, you read RX register, when you
write to this address you write into TX register.

TX and RX register are place to store/read data handled by UART. When UART
receive an new byte, interrupt is called and received byte is put into RX
register. Old byte is staying in RX register until next byte come.

By writing a new value into TX register, transmitting is triggered and byte
will be send. Due to full duplex character of UART peripherals, TX can be
active during RX.

\subsubsection{Register map}

All registers are listed in tables \ref{tab:uart_read_reg_map} and
\ref{tab:uart_write_reg_map}. Tables are split by type of operation
(read/write). Meaning of register during read operation is in table
\ref{tab:uart_read_reg_map} and during write operation is in table
\ref{tab:uart_write_reg_map}.

\begin{table}[h]
    \centering
    \begin{tabular}{|l|l|l|}
        \hline
        \textbf{Offset} & \textbf{Name} & \textbf{Purpose}            \\ \hline
        $+0$            & UDRx          & RX register of UARTx        \\ \hline
        $+1$            & UCRx          & Control register for UARTx. \\ \hline
    \end{tabular}
    \caption{UART register map - read operation}
    \label{tab:uart_read_reg_map}
\end{table}

\begin{table}[h]
    \centering
    \begin{tabular}{|l|l|l|}
        \hline
        \textbf{Offset} & \textbf{Name} & \textbf{Purpose}            \\ \hline
        $+0$            & UDRx          & TX register of UARTx        \\ \hline
        $+1$            & UCRx          & Control register for UARTx. \\ \hline
    \end{tabular}
    \caption{UART register map - write operation}
    \label{tab:uart_write_reg_map}
\end{table}

\subsubsection{Hacking}

UART peripherals does not have ability to change it parameters simply. They are
all hardwired. Anyway, you may want add more UART units, that is of course
possible.

\begin{lstlisting}[language=VHDL, frame=single]
entity uart is
    generic(
        BASE_ADDRESS: unsigned(23 downto 0) := x"000000"
    );
    port(
        clk: in std_logic;
        res: in std_logic;
        address: in unsigned(23 downto 0);
        data_mosi: in unsigned(31 downto 0);
        data_miso: out unsigned(31 downto 0);
        WR: in std_logic;
        RD: in std_logic;
        ack: out std_logic;
        --device
        rx: in std_logic;
        tx: out std_logic;
        rx_int: out std_logic;
        tx_int: out std_logic
    );
end entity uart;
\end{lstlisting}

Entity UART have same bus interface as all others modules. That mean clk, res,
address, data\_mosi, data\_miso, WR, RD, ack and BASE\_ADDRESS argument. But there
are also some device specific signals. These are rx, tx, rx\_int and tx\_int.

Signals tx\_int and rx\_int are interrupt requests. You should connect them to
the input of the interrupt controller. Signals rx and tx are transmitter and
receiver signals. You probably want connect them to the top levels pins. If rx
pin is not connected, there should be logical one on it.
