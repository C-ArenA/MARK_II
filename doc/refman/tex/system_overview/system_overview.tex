\section{System overview}

Heart of SoC is CPU. The CPU is only one master on the system bus and it control
all communication on this bus. To the CPU is directly connected interrupt
controller.

Bus consist of 24b address, 32b MOSI data, 32b MISO data, clock, reset, write,
read and ack signals. There is also 16b bus for interrupts.

All the peripherals are connected to the system bus, CPU select peripheral to
communicate with through address. Peripheral control ack signal to say to CPU
"data is ready for read" when CPU want read data, or "data is written" when CPU
want write data. This allow slower peripheral to cooperate with CPU, CPU simply
wait for them.

Some peripherals have special pins, these are commonly connected to the top
level entity pins, but there are few that are connected internally, for example
pll outputs.

SoC have multiple clock domain. There are five clock domains:

\begin{itemize}
	\item \textit{18,432 MHz} for UARTs
	\item \textit{22,5792 MHz} for audio (not implemented yet)
	\item \textit{100 MHz} for SDRAM driver
	\item \textit{100 MHz} for SDRAM itself (+30 degree phase shift)	
	\item \textit{25MHz} for CPU, VGA driver and reset of system
\end{itemize}

Due to used FPGA in board (10M25SAE144) all clocks are generated using 
external oscilators, and only clocks for SDRAM are generated using PLL.
