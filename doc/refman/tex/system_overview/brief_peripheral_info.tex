\subsection{Brief peripheral informations}

\subsubsection{On-chip ROM}

On-chip ROM is small memory for your program, it can be initialized with content
 at the synthesizing design into FPGA. This memory have only 1kB (256 words).

\subsubsection{GPIO}

Really simple peripheral but widely used. GPIO consist from two eight bit ports.
 Each pin in port can be configured as output or input.

\subsubsection{SysTimer}

Simple 24b timer. It will generate interrupt on compare match and then start
counting from zero.

\subsubsection{Interrupt driver}

Simple interrupt controller allowing disabling individual interrupts. Also take
care about priorities.

\subsubsection{UART}

Universal asynchronous receiver/transmitter is intended for serial communication
with many devices like computers, another micro-controllers, modules and so on.
It has build in baud rate generator and also provide configurable interrupt sources.

\subsubsection{Timers}

Basic 16b timers with PWM output generation and interrupts. Timer can generate
interrupt on compare match like SysTimer, on overflow or on both.

\subsubsection{On-chip RAM}

Bigger place for data and program. CPU can run program from there too because
it is designed as Von Neumann architecture. RAM have 4kB in size, so there are
1024 words. This RAM is faster than external, you can use it for accelerate your
programs by storing them here, or at least, store variables here.

\subsubsection{PS2 driver}

With PS2 driver you can connect keyboard to the SoC. This driver is really
simple and it is able only to receive data from keyboard. It will also generate
interrupt when data from keyboard come.

\subsubsection{VGA driver}

VGA driver is useful form of output. It work in text mode with resolution 80x30
characters. Each character have size 8x16 pixels. This is 640x480@73hz effective
resolution. VGA driver also support colors, you can specify color for foreground
(character) and for background.
