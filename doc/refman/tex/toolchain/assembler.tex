\subsection{Assembler}

Simple two pass assembler for MARK-II is written in python 2.7 and is tested
under Linux. Assembler have basic support for macros and conditional translation.
In combination with linker, this simple assembler is able to do everything you
need, for development.

\subsubsection{Preprocessor}

Assembler have build-in preprocessor. This preprocessor have one pass
architecture, due this it isn't able to perform any forward declarations and any
recursions in macros. Preprocessor have build-in following commands:

\begin{itemize}
    \item \textbf{\#define symbol} -
    Define an symbol that is used ONLY with conditional assembly.

    \item \textbf{\#ifdef symbol} -
    If symbol is defined, assembler will parse following code, otherwise code
    will be skipped to the nearest \#endif. Note: "\#if" block can be nested.

    \item \textbf{\#ifndef symbol} -
    Same as \#ifdef but assembly code when symbol does not exist.

    \item \textbf{\#endif} -
    Identify end of condition assembly block.

    \item \textbf{\#include file} -
    Include file into current buffer.

    \item \textbf{\#macro name parms} -
    Define an new macro, everithing up to \#endmacro will be body of new macro.

    \item \textbf{\#endmacro} -
    Close macro body and write it into macro table.
\end{itemize}

\subsubsection{Macros}

Macro is small sequence of code that can be simple defined once and pasted into
code many times. Lets say there is an macro called "setLed". You can invoke it
with code like this:

\begin{lstlisting}[language={[x86masm]Assembler}, frame=single]
    $setLed
\end{lstlisting}

Prefix \$ is mandatory when invoking macro and tell the preprocesor "I wan't
paste macro setLed here!". Preprocesor then take body of setLed macro and paste
it into output buffer.

Macro can be defined using \#macro and \#endmacro preprocesor commands. For
example we will define macro setLed.

\begin{lstlisting}[language={[x86masm]Assembler}, frame=single]
    #macro setLed
        MVIL R1 0x0001
        ST R1 PORTA
    #endmacro
\end{lstlisting}

Macros can have arguments in ther definition. This feature is used in same way
as argument in C functions. For first you have to define macro with argument in
name. Let's say we want an macro "send" for sending by urat. This macro will
have one argument - byte to send. Code should look something like this:

\begin{lstlisting}[language={[x86masm]Assembler}, frame=single]
    #macro send byte
        ;some code for prepare sending
        ST byte UDR0
    #endmacro
\end{lstlisting}

Well, now we have macro defined, we will use it somewhere in the code like this:

\begin{lstlisting}[language={[x86masm]Assembler}, frame=single]
    $send 0xAA
\end{lstlisting}

Value "0xAA" will be automatically placed everywhere "byte" is written, when
macro is invoking.

Assembler have build in support for labels inside macros. This mean, you can
define label in macro and then call it. But this feature have exception that
normal labels doesn't.

You can call label only within macro definition itself. There is no way to call
label defined inside macro from normal code. This is because when you invoke
macro, labels get automatically renamed. There is unique name for each instance
of macro.

See example bellow to get this clear.

\begin{lstlisting}[language={[x86masm]Assembler}, frame=single]
    ;define new macro for delay

    #macro delay time
        .MVI R1 time
        loop:
        DEC R1 R1
        CMP EQ R1 R0 R2
        BZ R2 loop
    #endmacro

    ; some useful code is here
    OR R0 R0 R0

    $delay 100      ; invoke macro "delay"

    ; assembler take definition of delay macro - replace loop "label" with "loop1"
    ; (1 because this is first instance of this macro) - and place macro body there

    ; there is another tons of usefull code
    OR R0 R0 R0

    $delay 1000     ; next instance of delay - now "loop2" is created
\end{lstlisting}

But macros also have some limitations. Mainly:

\begin{itemize}
    \item Macros can't be nested.
    \item Macro have to be complete defined before invoking.
    \begin{itemize}
        \item This also mean any recursion.
    \end{itemize}
\end{itemize}

\subsubsection{Numbers}

Numbers are parsed in C like form so:

\begin{itemize}
    \item \textbf{0x10} - hexadecimal number, 16 in decimal form
    \item \textbf{0b1010} - binary number, 9 in decimal form
    \item \textbf{74} - decimal number 74
\end{itemize}

\subsubsection{Labels}

When they are defined, they have to be followed with semicolon ':'. When used
(eg. with CALL instruction), semicolon is not used anymore. Example code:

\begin{lstlisting}[language={[x86masm]Assembler}, frame=single]
    halt:
        OR R0 R0 R0
        BZ R0 halt
\end{lstlisting}

\subsubsection{PseudoInstructions}

They have to start with dot '.'. They are used for controlling assembler. Following
PseudoInstructions are supported:

\begin{itemize}
    \item \textbf{.CONS name value} -
    Set constant <name> with contain <value>. Assembler simply replace all
    <name> in source code with <value>. This is not an macro and only numeric
    values can be processed!

    \item \textbf{.ORG value} -
    Set location counter to specified location during pass1.

    \item \textbf{.DAT word*} -
    Reserve space in memory at current location with size for all words.
    Memory is then filled with them. Every word have to be separated with
    space.

    \item \textbf{.DS size} -
    Only reserve space in memory at current location with size <size> in words.

    \item \textbf{.EXPORT name} -
    Export label for use in another object file. This pseudoinstruction create
    an special symbol. Special symbols are then printed into object file for
    linker.

    \item \textbf{.IMPORT name} -
    Import exported label from another object file. This pseudoinstruction
    create an special symbol. Special symbols are then printed into object file
    for linker.

    \item \textbf{.MVI register value} -
    Load register with 32b value. This is useful when you want load constants
    into registers.
\end{itemize}
