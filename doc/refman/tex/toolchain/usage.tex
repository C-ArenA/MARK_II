\subsection{Usage}

\subsubsection{Assembler}

\begin{lstlisting}[language=bash, frame=single]
    Example usage: assembler main.asm

    Arguments:
        -h --help           Print this help.
        -o --output         Output object file name.
           --skip-linker    Generate relocatable loader module
                            instead object file. Can be used for
                            skipping linker if linking is not needed.
           --version        Print version number and exit.
\end{lstlisting}

Assembler take one file with ".asm" and translate it into object file ".o".
Object files are then read by linker and linked together into loadable module
".ldm". You can skip linker step by specifying "--skip-linker" argument.

Please note, if you use preprocessor directive "\#include file.asm", you don't
need this file linked, because preprocessor take whole "file.asm" and paste it
into current file. Linker is needed when you used pseudo-instruction ".IMPORT
label".

\subsubsection{Linker}

\begin{lstlisting}[language=bash, frame=single]
    Example usage: linker example1.o example2.o

    Arguments:
        -h --help           Print this help.
        -o <file>           Output LDM file name. If not specified,
                            name of first object file will be used.
        -l <path>           Path to look for libraries.
           --version        Print version number and exit.
\end{lstlisting}

Link multiple object files into one loadable module. Linker is necessary when
you split your code into multiple files and use ".EXPORT label" and ".IMPORT
label" pseudo-instructions.

In that case just compile each file separated and then call:

\begin{lstlisting}[language=bash, frame=single]
    # linker file1.o file2.o
\end{lstlisting}

Linker is also needed when one want use compiled libraries. These 
libraries are classical object files without any modifications. Linker 
first try to link your ldm file from object files given as argument, 
when some missing labels are occurred, patch given with -l argument are 
searched for object files. 

\subsubsection{ldm2mif}

\begin{lstlisting}[language=bash, frame=single]
    Example usage: ldm2mif example.ldm

    Arguments:
        -h --help           Print this help.
        -o <file>           Output MIF name. If not specified name of
                            input file will be used.
        -r <address>        Relocate source. Add just immediate
                            addresses of these instructions that use
                            relative addressing using labels.
                            You have to specify <address> in hex
                            where the code will be stored. Default
                            value is 0x000000.
        -s <size>           Size of memory, default value is 8.
                            Memory range is from 0 to 2^<size>.
           --version        Print version number and exit.
\end{lstlisting}

This simple tool is used to convert loadable module into ".mif" files for
Quartus. You can specify size of memory and also base address of memory. All
relative symbols (like jump into labels) will be recalculated relative to the
base address.

\subsubsection{disassembler}

\begin{lstlisting}[language=bash, frame=single]
    Example usage: disassembler uart.ldm

    Arguments:
        -h --help           Print this help.
        -o --output         Output file name.
           --version        Print version number and exit.
\end{lstlisting}

Read compiled loadable module and translate it back to assembler source codes.

\subsubsection{loader}

\begin{lstlisting}[language=bash, frame=single]
    Example usage: loader -b 0x400 -p /dev/ttyUSB0 example.ldm

    Arguments:
        -h --help           Print this help.
        -b <address>        Base address, using hex C like syntax,
                            to store source.
                            Loader also perform relocation of the
                            given source to this address.
        -p <port>           Port where MARK-II is connected. For
                            example /dev/ttyUSB0.
           --baudrate       Set baudrate for port. Default value is
                            38400.
           --version        Print version number and exit.
        -e --emulator       Add this option if you are connecting
                            to emulator.
\end{lstlisting}

Simple tool for loading programs into CPU using UART. Simply specify the base
address where you want store your code, specify port where MARK-II is connected
and wait. MARK-II must be connected before sending started.

\subsubsection{emulator}

\begin{lstlisting}[language=bash, frame=single]
    Example usage: emulator -g -p /dev/pts/2 -r rom.mif

    Arguments:
        -h --help           Print this help and exit.
        -p --port           Device where uart0 will be connected.
                            Can be /dev/pts/2 for example.
        -r --rom            Filename of file that will be loaded
                            into rom0.
        -g --gui            Run with simple GUI.
           --version        Print version number and exit.
\end{lstlisting}

Great way to test your program and get in touch with MARK-II. Enable execution
of programs in almost same way like real HW. Also emulate uart0 so you can
connect serial port monitor.

\subsubsection{vbcc}

For information about ussage vbcc please refer original vbcc documentation that
can be found in directory /sw/vbcc/doc.
