\subsection{vbcc}

MARK II toolchain also contain ISO C Compiler. This compiler was wrote by
Dr. Volker Barthelmann and for full documentation please refer original vbcc
documentation that can be found in folder /sw/vbcc/doc.

Compiler can translate C programs into assembler sources. These sources then can
be translated into object files, linked together and loaded into MARK II memory.

Homepage of vbcc can be found at this link: \url{http://www.compilers.de/vbcc.html}.

Purpose of this section isn't give informations about vbcc usage, for this please
refer original documentation. Purpose of this section is to inform about register
usage, calling conventions and backend related things.

\subsubsection{Register usage}

CPU have sixteen registers, three of them are special registers. Almost all
of these registers are used for compiler purposes. Function of all registers
can be found in table \ref{tab:registers_list_ussage}.

\begin{table}[h]
    \centering
    \begin{tabular}{|l|l|l|l|}
        \hline
        \textbf{Register name} & \textbf{Purpose}  & \textbf{Register name} & \textbf{Purpose} \\ \hline
        R0                     & zero register     & R8                     & tmp register     \\ \hline
        R1                     & compiler reserved & R9                     & tmp register     \\ \hline
        R2                     & compiler reserved & R10                    & tmp register     \\ \hline
        R3                     & compiler reserved & R11                    & tmp register     \\ \hline
        R4                     & condition flag    & R12                    & tmp register     \\ \hline
        R5                     & return value      & R13                    & frame pointer    \\ \hline
        R6                     & tmp register      & R14                    & program counter  \\ \hline
        R7                     & tmp register      & R15                    & stack pointer    \\ \hline
    \end{tabular}
    \caption{Register usage}
    \label{tab:registers_list_ussage}
\end{table}

Compiler reserved registers $R1$ - $R3$ are used by compiler for loading
variables, calculating addresses, comparisons and so on.

Condition flag register $R4$ are used for conditional jumps. CMP instruction
will always store return value into this register. Following instruction should
be branch instruction that will use this register.

Registers $R1$ to $R4$ are always pushed into stack in function head and then
poped out at function bottom.

Register $R5$ is used as return register. When function have to return some value
this value will be returned in this register.

Frame pointer and stack pointer, registers $R13$ and $R15$ are used for manipulating
stack. At stack are stored all local variables. For more information please refer
section about stack usage.

Program counter is maintained by CPU and vbcc backend doesn't manipulate it directly.

There are seven tmp registers, registers $R6$ to $R12$. These registers can be
used freely by vbcc backend or by assembler programmer. But at head of each
function/subroutine, used registers have to be pushed into stack.

\subsubsection{Stack usage by functions}

When function is called, new stack frame is emitted. For this frame pointer and
stack pointer registers are used. Stack frame consist from arguments passed into
called function, return address and old frame pointer.

So whole calling sequence look like this:

\begin{lstlisting}[language={[markII]Assembler}, frame=single]
    PUSH R6     ;push two arguments into stack
    PUSH R7
    CALL foo    ;call function foo
\end{lstlisting}

Head of function foo will look like this:

\begin{lstlisting}[language={[markII]Assembler}, frame=single]
    .EXPORT foo
    foo:
        PUSH R13    ;store old frame pointer
        MOV SP R13  ;create new frame pointer

        ;make space for auto class variables by pushing R0

        ;store all used registers
\end{lstlisting}

At first, name of function is exported, then label is generated. In new function
first thing to do is backup frame pointer by pushing $R13$. Then creating a new
frame pointer by copy value from $SP$ into $R13$. Finally, function will store all
registers that will use with PUSH instruction.

At the end of function, return value is moved into $R5$ and function bottom is
generated.

Function bottom consist from poping out used registers from stack, restoring frame pointer,
and calling instruction RET. This look like this:

\begin{lstlisting}[language={[markII]Assembler}, frame=single]
    ; move return value into R5

    ; pop all used registers

    MOV R13 SP  ; restore SP
    POP R13     ; restore FP
    RET
\end{lstlisting}

Stack is also the place where all locals variables are stored. For their addressing
is used frame pointer. And they are stored after stored old frame pointer.

\subsubsection{Simple optimizations tips}

\begin{itemize}

    \item
    Reorder local variables declaration. First declared variable should be
    variable that is used most often. Second most often used variable should be
    declared right after first declaration. Order of next variables do not
    matter. Also make sure, that first and second variables are not arrays or
    structures.

    \item
    Avoid multiplication whenever is possible. Use shifts instead, this is
    always faster than multiplication.

\end{itemize}

\subsubsection{Libraries}

C Standard Library is not yet implemented but there is SPL - Standard
Peripheral Library for MARK II available. This library is only header files
defining some useful macros and constants for reading and writing registers,
accessing RAM, ROM and VRAM memories and bit mask for acessing various bits in
registers. For more details please refer SPL reference manual, or see examples
in sw directory.

Usage is really simple. When toolchain is installed, install script emit path to
SPL. These path have to be parsed into vbcc at compile time with -I argument.
Then you can normally include header file spl.h as usual.
