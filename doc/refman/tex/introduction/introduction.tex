\section{Introduction}

MARK II is simple SoC written in VHDL for synthesization into Altera Cyclone
FPGA. SoC consist of simple RISC like CPU and a few peripherals. There is also
full featured toolchain for writing programs that can run on MARK-II CPU.

CPU is 32 bits wide and following Von Neumann architecture, can address up to
$2^{24}$ words, has 16 interrupt sources, 16 registers including three special
ones.

To the CPU can be connected memory mapped peripherals, following 
peripherals are available: UART, PS/2, VGA driver, system timer, onchip 
RAM and ROM, interrupt controller and driver for external SDRAM.

Thanks to toolchain, you can write programs in human way. Toolchain consist of
Assembler, Linker, Emulator, Disassembler, Loader and ldm2mif converter. There is
also backend for vbcc so you can compile your C programs too.

This document is reference guide for MARK-II CPU. But because this project is really
complex and still it is one man show, some informations in this manual can be
outdated. If you are get stuck on something, please see source codes or
contact author.

MARK-II is home project, created just for fun. There isn't any main goals
except one, have fun from coding. As home fun project, is mostly licensed under
MIT license to be free for everybody to try run it. But due to complexity there
are some parts that is licensed under another license, for example vbcc front
end.

Anyway, some goals exist, but they are more like "I want to try ..." rather than
"I have to ...", these goals are changing in time and they are following topics
that I want to explore.
